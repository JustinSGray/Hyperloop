\documentclass[heading.tex]{subfiles} 
\begin{document}

\section{Introduction}

Hyperloop is a conceptual transportation system designed to lower costs and travel times relative to California’s current high-speed rail project.
\cite{Musk} Elon Musk and a team of engineers from Tesla Motors and the Space Exploration Technologies Corporation (SpaceX)
proposed the idea in August 2013, as an open design to be vetted and further refined through public contribution.
The concept deviates from existing high-speed rail designs by eliminating the rails,  enclosing the passenger pod in a tube under a partial vacuum,
and supporting the passenger pod on air bearings. Propulsion is handled by a set of linear electromagnetic accelerators mounted to the tube,
with the entire system being held above ground on concrete columns maintaining a straight trajectory.
Although Hyperloop is similar to other vacuum tube train (VacTrain) concepts \cite{ET3}, the soft vacuum represents a distinct difference.
It allows the train to run on air-bearings, thus removing the need for a magnetic levitation system used on the other VacTrain concepts.
The air bearings require a source of pressurized air, which is provided by an electric compressor system powered by on-board batteries.
The compression system also helps the pod approach sonic speeds, by reducing the amount of air being forced around the sides of the vehicle.

\begin{figure}[hbtp]
\centering
\includegraphics[width=\textwidth]{images/hyperloopAlphaSketch.jpg}
 \caption[Hyperloop Concept Sketch]{Hyperloop-alpha concept sketch of the passenger pod.}
\label{f:hyperloopSketch}
\end{figure}


Musk’s original Hyperloop proposal included individual high-level analyses of many major subsystems including the pod compression system,
elevated support structure, and propulsion system. While this demonstrated the basic viability of the concept, it did not address
significant system level trade-offs that cascade into multiple subsystems. The thermal management and aerodynamic constraints need
to be better understood before ultimately approaching the problem with the dual objectives of minimizing ticket cost and travel time.
The modeling approach described in this paper relies on the notion that the compression subsystem on the passenger pod is similar to the
compressors on a turbine engine. They both operate at higher Mach numbers and low-pressure conditions. Therefore, experiences drawn
from aircraft engine design were adapted to address three overarching system challenges omitted from the original proposal.
The first section identifies a more stringent coupling between passenger pod geometry and tube size, the second revisits compressor
analysis and battery sizing, and the final section addresses the requirements driven by thermal interactions. The results show that
the Hyperloop concept is feasible, but certain estimates from the original proposal may be been overly optimistic due to each discipline
being approached independently.

\begin{figure}[hbtp]
\centering
\includegraphics[width=\textwidth]{images/hyperloop_cad.png}
 \caption[Hyperloop geometry assembled in OpenCSM]{Calculated baseline inlet (left), nozzle (right), and full assembly (bottom) for a pod speed of Mach 0.8. Geometry rendered in OpenCSM, a parametric solid modeling tool.}
\label{f:hyperloopCAD}
\end{figure}

\section{Baseline Design and Subsystem Modeling Theory}

Due to the novelty of the Hyperloop design, the stack up of many subsystem uncertainties makes even the highest level trends difficult to pin down.
The initial subsystem modeling provides some concrete baseline numbers and better identifies the underlying details driving
the overall design.

\subsection{Tube Diameter}

The pod travels through a fixed diameter tube displacing air around itself. The air reaches the
pod at a relative velocity equal to the pod speed and accelerates as it is forced through the smaller bypass area around the pod.
Assuming a circular cross section for both the pod and tube, then the bypass air must travel through an area given by

\nomenclature{A}{Area (m^{2})}
\nomenclature{r}{Radius (m)}
\begin{equation*}
A_{bypass} = \pi(r_{tube}^2-r_{pod}^2)
\end{equation*}
\nomenclature{\rho_}{Density (\frac{kg}{m^{3}})}
\nomenclature{\dot{W}, \dot{m}}{Mass Flow Rate (kg/s)}
Since $A_{tube}$ and the air density $\rho_{air}$ within the tube are both constant for given tube size, temperature and pressure, the mass flow rate of
the air traveling around the pod ($\dot{W}_{bypass}$) grows linearly with the velocity of the pod.

\nomenclature{V}{Velocity ($\frac{m}{s}$)}
\begin{equation*}
\dot{W}_{bypass} = \rho_{air} A_{tube} V_{pod}
\end{equation*}

However, the flow reaches a physical limitation as it approaches the speed of sound. At this point, no additional flow can escape
around the sides of the vehicle without increasing the density of the air.
For a given area ratio, the limiting pod speed can be determined based on isentropic flow relations, where $\gamma$ is the heat capacity ratio.

\nomenclature{\gamma}{Heat Capacity Ratio}
\nomenclature{MN}{Mach Number}
\begin{equation*}
\frac{A_{tube}}{A_{bypass}} = \left(\frac{\gamma+1}{2}\right)^{-\frac{\gamma+1}{2\left(\gamma-1\right)}}\frac{\left(1+\frac{\gamma-1}{2}MN^{2}\right)^{\frac{\gamma+1}{2\left(\gamma-1\right)}}}{MN}
\end{equation*}


If the bypass air velocity, expressed as a Mach Number ($M$) above, reaches the speed of sound then the pod will act like a piston in a tube;
increasing the air pressure in front and lowering the pressure behind it. Without a ducted pod the limiting speed is around
120 $\frac{m}{s}$, or Mach 0.3, for the tube and pod size in the original proposal. This limit is shown by the intersection of the blue lines at the
bottom of figure \ref{f:flowLIMIT}. Such low speeds would not allow the Hyperloop concept to significantly reduce travel times between Los Angeles
and San Francisco relative to high speed rail. This limited pod speed is strongly dependent on the ratio of the pod cross-sectional area to the tube
cross-sectional area.

\begin{figure}[hbtp]
\centering
\includegraphics[width=\textwidth]{images/tube_flow_limit3.png}
\caption{Hyperloop speed limits as a function of three different area ratios}
\label{f:flowLIMIT}
\end{figure}


To further increase speeds, an inlet and compression system is needed to help draw additional air through the pod. 
The vertical distances between the choked flow limit (dashed lines) and the required flow (solid lines) of figure \ref{f:flowLIMIT}
signifies the minimum compressor flow rates necessary to achieve a given speed.  Increasing the area ratio to 7.5 yields a max speed 
slightly below Mach 0.7.

\begin{figure}[hbtp]
\centering
\includegraphics[width=\textwidth]{images/diffused_inlet.png}
\caption[Inlet Diffuser]{Vehicle speed determines amount of diffusion necessary}
\label{f:diffuse}
\end{figure}


Additional non-linearity is introduced when considering the limitations of the compression system. Increasing the demands on the compressors
causes the required pod inlet diameter to grow in order to handle the increased flow. As depicted in figure \ref{f:diffuse}, the area at the front of
the inlet ($A_{tube,c}$) must be smaller than the area at the front of the first compressor ($A_{diffused,c}$). This expansion is necessary to slow
the air down to manageable speeds before entering the first compressor. Traveling faster results in larger mass flow requirements,
which drives the pod diameter up, subsequently changing the area ratio and further increasing the mass flow requirements.
This compounding relationship between 
speed and tube diameter is shown in figure \ref{f:machRAD}. For each point on this graph, the full system model converges on the minimal
possible tube diameter, given a desired pod Mach number. Figure \ref{f:machRAD} also accounts for compressor performance limitations and
the steady-state operating tube temperature that reaches quiescence above ambient conditions. The family of curves represents the
improvements that could be gained from designing a primary compressor capable of handling incoming air moving at higher mach numbers denoted by c1MN.

\nomenclature{c1MN}{Mach number at the front face of the first compressor}

\begin{figure}[H]
\centering
\includegraphics[width=\textwidth]{images/mach_vs_rad4.png}
\caption[Tube and Pod Radius vs Mach]{Exponential relationship between pod speed and required tube radius, for three diffused mach numbers}
\label{f:machRAD}
\end{figure}

The exponential relationship between pod max speed and required tube radius explicitly shows the trend depicted by
connecting the dots of figure \ref{f:flowLIMIT}. The absolute tube radius is charted rather than an area ratio for visual simplicity
and as a more of a direct indicator of total system cost. It's worth noting that the pod size is also free to vary at each converged point
and is plotted on the cyan line. Somewhat surprisingly, the pod radius remains
nearly constant for every valid solution at each targeted speed condition and for every diffused mach design.
This can be attributed the extremely sensitive nature of the inlet
size when perturbed in either direction. Decreasing the inlet diameter adversely effects pod air intake more-so than it helps
bypass air throughput; and increasing the inlet diameter harms the bypass area more than it improves
compressor intake. The inlet size is also somewhat governed by the
necessary bearing flow requirements. This plot accounts for
the pod compression system and is therefore shifted further right than if the intersection points from figure \ref{f:flowLIMIT}
were translated and directly overlaid. 

The data shows that above Mach .85, the minimum allowable tube size gets very sensitive to pod travel speed. This indicates that speeds
greater than Mach .9 are likely not feasible. Even at a Mach .8, the tube still needs to be approximately 4 meters in diameter, roughly
twice the size considered in the original proposal. With a maximum speed of Mach .8, the travel time is greater than 45 minutes and
additional time would also be necessary to reduce acceleration around banked turns. Although this performance is much less favorable than the performance
described in the original proposal, it still represents an improvement over what can be achieved with a high speed rail.


\subsection{Compression System}

The compression system serves dual purposes. It provide a means of exceeding the nominal Kantrowitz limit
and also supplies pressurized air to support the air bearing system.
Each of these functions requires a minimum airflow to rise to a specific pressure gain,
which combine to define the total airflow requirements for the whole sub-system.
The system is comprised of an inlet, two compressors, two heat exchangers, a nozzle, and multiple ducts leading to air bearings.

\begin{figure}[hbtp]
\centering
\includegraphics[width=\textwidth]{images/compressor_schematic.png}
\caption{Schematic flow diagram of the modeled compressor system}
\end{figure}

The compression system is modeled as a one dimensional cycle, representing components as thermodynamic processes that are
subsequently chained together. Each component is responsible for calculating gas properties across its boundaries and appropriately
enforcing conservation equations across the entire system. The subsystem leverages the open-source Cantera package, which is 
responsible for calculating chemical kinetics, thermodynamics, and transport processes. These models predict the instantaneous
power consumption of the compressors, temperature rise, and upstream conditions necessary to supply sufficient airflow to the bearings.
These power requirements are a function of the chosen cycle and are both affected by and contribute to the thermal conditions. 
Combined with the estimated travel time, these requirements impact battery sizing and coolant storage requirements.

\subsection{Battery Sizing}

The battery power requirement was estimated based on the calculated max pod speed and compressor power requirements. Using the
notional velocity profile described in the original proposal (and included in \cref{app:route}) the average velocity can be
estimated by normalizing the profile then scaling it based on the the recalculated top speed. The total mission duration is then
calculated based on the average speed and tube length. The resulting time estimate multiplied by the instantaneous compressor
power requirements and a 30\% safety margin results in a overall battery storage requirement.

The necessary on-board battery size was found to be inversely related to max pod speed, as show in figure \ref{f:battery}.
This indicates that compressor power requirements are less sensitive to increased speeds than total trip time. This sensitivity
is highly dependent on the velocity profile; reducing the time spent at max speed could result in a positive correlation between
battery size and pod speed. It should be noted that the Y-axis of figure \ref{f:battery} begins at 340kW for visual simplicity.
The overall reduction in battery size is on the order of 25\% if speed is increased from Mach 0.7 to Mach 0.9. Additional mission
analysis details can be found in \cref{app:route} and work done by Mathworks \cite{Rouleau}.

\begin{figure}[hbtp]
\centering
\includegraphics[width=\textwidth]{images/mach_vs_energy.png}
\caption[Battery requirements as a function of pod speed]{Battery requirements as a function of pod speed. Note: the y-axis begins at 340kW-hr.}
\label{f:battery}
\end{figure}

\subsection{Tube Temperature}

As each pod passes through the tube, it adds energy to the air in the form of heat. Considering the frequency of the continuous operating cycle,
this could potentially heat the overall Hyperloop system to excessive temperatures.
To combat this effect, the original proposal recommends a heat exchanger system that would 
be integrated into the compression system. These intercoolers would use water stored in on-board tanks to cool the
air and assist secondary compression. The resulting steam could then be stored in a separate tank and offloaded once the pod reached its destination.
However, initial calculations show that using water for cooling is not an ideal design for two reasons:

1) The flow rate of water needed to remove the heat added by the compressors is very large, and the sheer volume constraints of storing
the resulting steam would outweigh the benefits.

2) The heat addition from each pod compressor cycle is fairly low relative to other heat transfer mechanisms occurring between the Hyperloop
and the external environment. Even without an active on-board cooling solution, the tube temperature would be dominated by other factors.

The following two sections provide additional details about the engineering models used to draw these conclusions.

\subsubsection{Pod Cooling Requirements}

The limits and requirements of a hypothetical on-board heat exchanger can be estimated with a straightforward energy balance. The
effectiveness of a heat exchanger can be described as the ratio of actual heat transfer over the maximum possible heat transfer.

\nomenclature{Q}{Heat flow rate (W)}
\begin{equation*}
{Q}_{released}  = effectiveness * {Q}_{max}
\end{equation*}


with $Qmax=\left(T_{hot,in} - T_{cold,in}\right) [ \dot{m} C_{p} ]_{fluid}$ where each fluid has a $\dot{m} C_{p}$ and the fluid with the lowest
product is used to determine the maximum heat transfer. In order to satisfy the energy balance $Q_{released}=Q_{absorbed}$ the following must be true,

\nomenclature{\dot{m}}{Mass flow rate}
\nomenclature{T}{Temperature (K)}
\nomenclature{C_{p}}{Heat capacity at constant pressure ($\frac{J}{kg-K}$) }
\begin{equation*}
\dot{m}_{air} C_{p, air} (T_{out, air} - T_{in, air}) = {Q}_{released} = {Q}_{absorbed}= \dot{m}_{water} C_{p,water} (T_{out, water} - T_{in, water})
\end{equation*}

where the $T_{out}$  of each fluid is unknown. With assumed mass-flow rates and initial temperatures, a valid combination of Tout‘s of
each fluid can be found through solver iteration. Valid effectiveness levels for heat exchangers can be estimated based on the E- NTU
method. The effectiveness for a counter flow heat exchanger with a $\frac{C_{min}}{C_{max}}$ of ~0.25 was chosen with air and water as the working fluids. 
The following conditions satisfied an energy balance with an extremely optimistically assumed effectiveness of 0.9765, and the proposed requirement to fully
cool the air back down to inlet temperatures.

\begin{table} [H]
\centering
\begin{tabular}{|c|c|c|c|c|c|c|}
\hline 
Fluid & Cp & Tin & Tout & mdot & Q (kJ/s) & Qmax \\ 
\hline 
Air & 1.006 kJ/kh-K & 791 K & 300 K & 0.49 kg/s & -242 & 247.9 \\ 
\hline 
Water & 4.186 kJ/kg-K & 288.15 & 416.6 K  & 0.45 kg/s & 242 & 247.9 \\ 
\hline 
\end{tabular} 
 \caption{Heat Exchanger Fluid Properties}
\end{table}

With a 35 minute trip, $0.45 \frac{kg}{s} * 60 \frac{s}{min} * 35min = 945 kg$ of standard temperature/pressure water would need to be carried 
with steam tanks over a hundred meters in length. This doesn't even account for the second stage heat exchanger, making the system nearly infeasible
with water and unpressurized tanks. Various systems involving alternate coolants such as liquid air or pressurized tanks could be explored.

Further discussion of heat exchanger sizing can be found in the heat exchanger section of the appendix.
The calculations explore the possibility of multi-pass heat exchangers
and the logarithmic mean temperature difference (LMTD) of the heat exchanger is considered.
\cite{Cengal}
\cite{Turns}


\subsubsection{Equilibrium Tube Temperature}
A high-level assessment of the overall steady-state heat transfer between the 300 mile Hyperloop tube and the ambient atmosphere is
also investigated. The outer diameter of the pipe is chosen as the control surface boundary used in the heat balance. Heat added from the pod exhaust
air and solar flux are considered the primary drivers for heat absorption into the tube. Conversely, heat released from the tube is modeled by means of
ambient natural convection and radiation out from the stainless-steel surface. The thermal interaction between the rarified internal air and
tube is not modeled.

The heat being added by the pods can be determined from the cycle analysis, or based purely on inlet total temperatures with isentropic
flow relations.

\nomenclature{P}{Pressure ($\frac{N}{m^{2}}$)}
\nomenclature{PR}{Pressure Ratio}
\nomenclature{PR}{Pressure Ratio}
\nomenclature{{\eta}_{adiabatic}}{Adiabatic Efficiency}
\begin{equation*}
T_{t} = T_{s} * [1 + \frac{\gamma -1}{2} MN^2]
\end{equation*}
\begin{equation*}
P_{t} = P_{s} * (\frac{ T_{t}}{T_{s}})^(\frac{\gamma}{\gamma -1})
\end{equation*}
\begin{equation*}
P_{t,exit} = P_{t,inlet} * PR
\end{equation*}
\begin{equation*}
T_{t,exit} = T_{t,inlet} + \frac{([T_{t,inlet}*PR^{(\frac{\gamma-1}{\gamma})}] - T_{t,inlet})}  {{\eta}_{adiabatic}}
\end{equation*}

Where PR is the compressor pressure ratio, MN is the mach number,  $\gamma$ is the specific heat ratio, and  ${\eta}_{adiabatic}$ is the
adiabatic efficiency.

With the air flow rate known, the heat flow rate per pod is obtained,

\begin{equation*}
{Q}_{pod}= \dot{m}_{air} C_{p,air} (T_{out, air} - T_{tube})
\end{equation*}
The peak heating rate from the pods scales linearly.
\begin{equation*}
{Q}_{peak}= Q_{pod} (\# ofpods)
\end{equation*}
The solar heat flow per unit area can be approximated, given the solar reflectance index (SRI) of stainless steel, non-normal incidence factor
of the cylinder and solar insulation (SIF).

\nomenclature{SRI}{Solar Reflectance Index}
\nomenclature{SIF}{Solar Insulation Factor}
\nomenclature{{\theta}_{nni}}{Non-normal Incidence factor (rad)}
\nomenclature{L}{Length (m)}
\nomenclature{OD}{Outer diameter (m)}
\nomenclature{\epsilon}{Emissivity Factor}
\nomenclature{\sigma}{Stefan-Boltzmann constant ($\frac{W}{m-K}$)}
\nomenclature{P_{rad}}{Radiated Power (W)}
\nomenclature{L}{Length}

\begin{equation*}
Solar = (1-SRI) {\theta}_{nni} SIF
\end{equation*}
Multiplying this by the viewing area of the tube (assuming no shade and constant sun)
\begin{equation*}
Q_{solar} = Solar * A_{view} = Solar * L_{tube} * OD_{tube}
\end{equation*}
Tube cooling can be attributed to two general mechanisms, radiation and natural convection. Radiation power per unit area can be
approximated to
\begin{equation*}
\frac{P_{rad}}{A} = \epsilon \sigma (T_{pipe}^4 - T_{ambient}^4)
\end{equation*}
where  $\epsilon$ is the emissivity factor and  $\sigma$ is the Stefan-Boltzmann constant.

Multiplying by the surface area of the tube, the total heating rate can be found,
\begin{equation*}
Q_{rad} =  \frac{P_{rad}}{A} * \pi L_{tube} OD_{tube}
\end{equation*}

Assuming the worst case scenario of no cross wind, convection is primarily driven by temperature gradients. The non-dimensional relation
between buoyancy and viscosity driven flows is parameterized using the following empirical constants. \cite{Berton} \cite{Incropera}

if 150 K $<  T_{amb} <$ 400 K:

\nomenclature{g}{Acceleration of gravity, 9.81 ($\frac{m}{s^{2}}$)}
\nomenclature{\beta}{Volume coefficient of expansion (K)}
\nomenclature{\upsilon}{Kinematic Viscosity ($\frac{m^{2}}{s}$)}
\nomenclature{Pr}{Prandtl Number, $\frac{\upsilon}{\alpha}$}
\nomenclature{Gr}{Grashof Number, $\frac{ g \beta \delta TL^{3}}{v^{2}}$}
\nomenclature{Ra}{Rayleigh Number, $\frac{\rho U_{\infty} L}{\mu}$}
\nomenclature{Nu}{Nusselt Number, $\frac{hL}{k}$}
\nomenclature{h}{Heat transfer coefficient ($\frac{W}{m^{2}-K}$)}
\nomenclature{k}{Thermal Conductivity ($\frac{W}{m-K}$) }

\begin{equation*}
\frac{g \beta T} {\upsilon^2} =  4.178\times10^{19} \times T_{amb}^{-4.639}
\end{equation*}

\begin{equation*}
Pr = 1.23 T_{amb}^{-0.09685}
\end{equation*}

if 400 K $<  T_{amb}  <$ 2100 K:


\begin{equation*}
\frac{g \beta T} {\upsilon^2}  = 4.985\times10^{18} \times T_{amb}^{-4.284}
\end{equation*}
\begin{equation*}
Pr = 0.59 T_{amb}^{0.0239}
\end{equation*}
The Grashof Number can then be approximated,


\begin{equation*}
Gr = \frac{g \beta T} {\upsilon^2}  (T_{tube}-T_{amb}) {OD}_{tube}^3
\end{equation*}
The non-dimensional Rayleigh number can then be calculated to estimate buoyancy effects, leading to the Nusselt number.


\begin{equation*}
Ra = Gr * Pr
\end{equation*}
\begin{equation*}
Nu = \Bigg(0.6 + \frac{0.387Ra^{\frac{1}{6}}}{[1+(\frac{0.559}{Pr})^{\frac{9}{16}}]^{\frac{8}{27}}}\Bigg)^2
\end{equation*}

From this point the total heat transfer from natural convection can be obtained,

\begin{equation*}
Q_{nat. conv} = hA \Delta T = \frac{k*Nu}{ {OD}_{tube}} \pi {L}_{tube} {OD}_{tube} (T_{tube}-T_{amb})
\end{equation*}

The steady state tube temperature can be found by varying the tube temperature until the rate of heat being released from the tube
matches the rate of heat being absorbed by the tube. Assuming numerous state variables provided in the source code, a steady state temperature of 120 F
was reached. This result suggests that there is no need for on-board cooling, however many baseline assumptions have yet to be validated or even defined.

\section{Full Model Integration}

Each subsystem contains variables pertinent to other disciplines, resulting in a large inter-dependence between components. 
This coupling requires a workflow with numerous sub-iterations in order to arrive at a physically valid state for the entire system.
At the highest level, a solver is necessary to perturb numerous design and state variables to characterize the system of equations and drive each variable
towards a specified valid criteria. For this task, the OpenMDAO framework provides numerous solvers and optimizers and simplifies the process of assembling
disparate codes into a cohesive workflow. The fully assembled Hyperloop model is broken down into 5 major assemblies as described below. 

\begin{enumerate}
  \item Compression System (\texttt{compress}): Performance and power consumption of the compressors
  \item Mission Analysis (\texttt{mission}): Estimate of travel time
  \item Pod Geometry (\texttt{pod}): Physical dimensions of the pod and calculations that depend on them
  \item Tube Flow Limitations (\texttt{flow\_limit}): Pod speed limitations based on choked flow restrictions between the pod and the tube
  \item Tube Wall Temperature (\texttt{tube\_wall\_temp}): Equilibrium temperature of the tube wall
\end{enumerate}

The connectivity between the assemblies is visually represented using XDSM charts below. Each green box represents a system that takes a
set of inputs and operates on them to produce a set of outputs. The connection of one system's outputs to another system's inputs is
visualized as a grey line. The cascading parallelograms indicate intersections, implying that an output is connected to multiple inputs. 


\begin{figure}[hbtp]
\centering
\includegraphics[width=\textwidth]{images/hyperloop_assembly_xdsm.png}
\caption{Hyperloop Top Level Assembly XDSM}
\label{f:hyperloopXDSM}
\end{figure}

The compression system and pod geometry systems in figure \ref{f:hyperloopXDSM} are further expanded into sub-assemblies as shown in figure \ref{f:podXDSM} and \ref{f:compressorXDSM}.

\begin{figure}[hbtp]
\centering
\includegraphics[width=\textwidth]{images/pod_assembly_xdsm.png}
\caption{Expanded \texttt{pod} assembly XDSM}
\label{f:podXDSM}
\end{figure}

\begin{figure}[hbtp]
\centering
\includegraphics[width=\textwidth]{images/compress_assembly_xdsm.png}
\caption{Expanded \texttt{compress} assembly XDSM}
\label{f:compressorXDSM}
\end{figure}

\subsection{Design Variables}
The system design is defined at the top-most level with the following design variables. The variables listed in \ref{t:designVars}
can be actively varied by the optimizer to achieve user defined objectives:

\begin{table} [H]
\centering
\begin{tabular}{|c|c|}
\hline 
Variable & Description\\ 
\hline 
\texttt{Mach\_bypass} & Mach number of the air traveling around the pod \\ 
\hline 
\texttt{Mach\_pod\_max} & Maximum travel Mach number of the pod \\ 
\hline 
\texttt{Mach\_c1\_in} & Mach number of the air at the back of the inlet \\ 
\hline 
\texttt{Ps\_tube} & Static pressure of the air in the tube \\ 
\hline 
\texttt{c1\_PR\_des} & First compressor pressure ratio \\ 
\hline 
\end{tabular}
 \caption[Design Variables]{Top-level Design Variables.}
 \label{t:designVars}
\end{table}

\subsection{Model Parameters}
The variables listed in table \ref{t:modelVars} are free for the user to set, but are not intended for the optimizer. Rather, they are general traits of the system.

\begin{table} [H]
\centering
\begin{tabular}{|c|c|c|c|c|c|}
\hline 
Variable & Description  \\ 
\hline 
\texttt{pwr\_marg} & Safety factor applied to the power req. for the pod \\ 
\hline 
\texttt{solar\_heating\_factor} & Fraction of solar radiation to consider in tube temp.\\ 
\hline 
\texttt{tube\_length} & Length of the tube  \\ 
\hline 
\texttt{n\_rows} & Number of rows of seats in the passenger pod  \\ 
\hline 
\texttt{length\_row} & Length allotted to each row of seats  \\ 
\hline 
\texttt{coef\_drag} & Drag Coefficient for the pod  \\ 
\hline 
\texttt{hub\_to\_tip} & Hub radius to tip radius ratio for the first compressor \\ 
\hline 
\end{tabular} 
 \caption[Model Parameters]{Top-level model parameters}
 \label{t:modelVars}
\end{table}

\subsection{Constraints and State Variables}

The Hyperloop concept presents a multidisciplinary design problem and encapsulating complexity into systems makes the relationship
between variables easier to understand.  Any connection originating from the left side of a component in an XDSM diagram represents a
cyclic connection. The cyclic connections in figure \ref{f:hyperloopXDSM}  and \ref{f:compressorXDSM} represent the coupling relationships
between the sub-systems. These couplings enforce a set of equality constraints that must be satisfied for any physically valid Hyperloop design. 

\begin{equation} \label{eq:flow}
	0.01*(\texttt{compress.W\_in} - \texttt{flow\_limit.W\_excess}) = 0
\end{equation}
\begin{equation} \label{eq:bearing}
	\texttt{compress.Ps\_bearing\_residual} = 0
\end{equation}
\begin{equation} \label{eq:temp}
	\texttt{tube\_wall\_temp.ss\_temp\_residual} = 0
\end{equation}
\begin{equation} \label{eq:area}
	0.01*(\texttt{pod.area\_compressor\_bypass} - \texttt{compress.area\_c1\_out}) = 0
\end{equation}

For constraints \ref{eq:flow} and \ref{eq:area}, a multiplier has been applied to the constraint to scale it for improved numerical convergence.
The first equality constraint ensures the pod compressor achieves the necessary flow intake given a maximum bypass flow.
The second enforces a minimum static air pressure for the air bearings. The third constraint drives the operating tube temperature to balanced conditions,
and the final constraint requires agreement between the pod geometry code and the compression system computed areas.

The optimizer varies  the following state variables to to satisfy the constraints. 

\begin{enumerate}
\item \texttt{compress.W\_in}
\item \texttt{compress.c2\_PR\_des}
\item (\texttt{compress.Ts\_tube}, \texttt{flow\_limit.Ts\_tube}, \texttt{tube\_wall\_temp.temp\_boundary}) \label{Temp}
\item (\texttt{flow\_limit.radius\_tube}, \texttt{pod.radius\_tube\_inner}) \label{Radius}
\end{enumerate}

State variables \ref{Temp} and \ref{Radius} are given as a list of variables signifying that they are linked variables with equivalent values. They are
treated as a single variable for the purposes of converging the model, but remain as distinct variables within each subsystem.

\subsection{Outputs}
There are a number of top-level output values that are of interest. They are listed in table \ref{t:outVars}.

\begin{table} [H]
\centering
\begin{tabular}{|c|c|c|c|c|c|}
\hline 
Variable & Description  \\ 
\hline 
\texttt{pod.radius\_inlet\_outer} & Overall pod radius \\ 
\hline 
\texttt{compress.W\_in} & Total mass flow through the compression system\\ 
\hline 
\texttt{compress.pwr\_req} & Total power required to drive the compression system  \\ 
\hline 
\texttt{mission.energy} & Total energy needed to power the compression system for one trip  \\ 
\hline 
\texttt{mission.time} & Travel time for one trip  \\ 
\hline 
\texttt{compress.speed\_max} & Maximum speed  \\ 
\hline 
\texttt{tube\_wall\_temp.temp} & Equilibrium tube temperature \\ 
\hline 
\end{tabular} 
 \caption[Model Outputs]{Top-level model outputs}
 \label{t:outVars}
\end{table}


Note: For a more complete design process, the values of the design variables would be optimized to minimize a combination of total power
consumption and travel time. However, the model does not currently calculate some key values needed to get a useful answer. In particular,
the linear accelerator and vacuum pump power needs to be modeled before a design optimization can be attempted.
It should be noted that you can select designs that are not realistic, particularly with respect to \texttt{pod.radius\_inlet\_outer}. There is
a strong relationship between \texttt{Mach\_pod\_max} and the \texttt{pod.radius\_inlet\_outer}. If you select a high 
\texttt{Mach\_pod\_max} (Above .8), you may find that the radius has shrunk below what is physically feasible without significant design
changes.


\section{Conclusions}
For the most part, the ideas and numbers given in the original Hyperloop proposal hold up using this analysis. However, the data shows that
there are two major changes to the design that need to be considered.

The tube cross section will need to be significantly larger than the original proposal. In the original proposal, the tube was sized with a diameter 2.23
meters. However, it appears that it will need to have a diameter closer to 4 meters to reach speeds maximum speeds of Mach 0.8.
On-board water based inter-coolers also seem impractical due to both volume and weight constraints. This may prove to be a non-issue since
temperature rise due to compression is significant less than originally estimated and only leads to a modest rise in steady-state tube
temperature. Assuming the tube was left uncovered, the heat rate from solar radiation would be an order of magnitude larger than the heat
rate added from pod compression systems. Further assuming a 90 degF day, radiation and convection out of the tube would lead to a
manageable steady state tube wall temperature of 120 degF.

Current tool-sets provide sufficient complexity management, however a larger breadth of disciplines need to be modeled before ultimately understanding
the feasibility of this unique multidisciplinary problem.

%\begin{thebibliography}{WoodTP}
%\bibitem[WoodTP]{woodTP} Wood, W.A., ``Multidimensional Upwind
 % Fluctuation Splitting Scheme with Mesh Adaption for Hypersonic Viscous
 % Flow,'' NASA/TP 2002-211640, Apr.~2002.
%\end{thebibliography}
%{\em Note that this entry is not necessarily in the correct NASA
%  format. Consult Technical Editing for correct reference format.}
\end{document}
